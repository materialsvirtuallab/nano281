\documentclass[aspectratio=169]{beamer}
\usepackage[utf8]{inputenc}
\usepackage{hyperref}
\usepackage{amsmath,amsfonts,amsthm,bm}
\usepackage{color}
\usepackage{graphicx} % Allows including images
\usepackage{booktabs} % Allows the use of \toprule, \midrule and \bottomrule in tables
\usepackage{tikz}
\usepackage{pgfplots}
\usepackage{listings}
\usepackage{courier}

\lstset{ %
  basicstyle=\scriptsize\ttfamily, % fonts that are used for the code
  breakatwhitespace=false,         % sets if automatic breaks should only happen at whitespace
  %breaklines=true,                 % sets automatic line breaking
  %captionpos=b,                    % sets the caption-position to bottom
  commentstyle=\color{gray}\textit,    % comment style
  keepspaces=true,                 % keeps spaces in text, useful for keeping indentation of code (possibly needs columns=flexible)
  keywordstyle=\color{blue},       % keyword style
  language=Python,                 % the language of the code
  %otherkeywords={*,...},          % if you want to add more keywords to the set
  rulecolor=\color{black},         % if not set, the frame-color may be changed on line-breaks within not-black text (e.g. comments (green here))
  showspaces=false,                % show spaces everywhere adding particular underscores; it overrides 'showstringspaces'
  showstringspaces=false,          % underline spaces within strings only
  showtabs=false,                  % show tabs within strings adding particular underscores
  stringstyle=\color{red}, % string literal style
  tabsize=4,	                   % sets default tabsize to 2 spaces
  columns=fixed                    % Using fixed column width (for e.g. nice alignment)
}

\hypersetup{
    colorlinks=true,
    linkcolor=red,
    filecolor=magenta,
    urlcolor=red,
}

\DeclareMathOperator*{\argmax}{argmax}
\DeclareMathOperator*{\argmin}{argmin}
\let \vec \mathbf

\newcommand{\classname}{NANOx81R Data Science in Materials Science}

\usetheme{metropolis}

\setbeamerfont{frametitle}{size=\Large,series=\bfseries}
\metroset{block=fill}

\title[\classname Course Admin]{\classname\\Course Admin}

\author{Shyue Ping Ong}
\institute[UCSD]{Aiiso Yufeng Li Family Department of Chemical and Nano Engineering\\
University of California, San Diego\\\url{http://materialsvirtuallab.org}}
\date{}

\begin{document}


\begin{frame}
    \titlepage % Print the title page as the first slide
\end{frame}


\begin{frame}{Course Objectives}
\huge{Provide a comprehensive introduction into the application of data science to materials science.}
\end{frame}


\begin{frame}{What will you learn in this course?}
    \begin{itemize}
    \item Understand key data science methods and how they relate to materials science.
    \item Apply best practices using real-world examples.
    \item Use Python and open-source libraries for practical analysis.
    \end{itemize}
\end{frame}


\begin{frame}{What this course is NOT}
    \begin{itemize}
        \item A probability and statistics course.
        \item A replacement for a rigorous data science course.
        \item A replacement for a materials science course.
    \end{itemize}
\end{frame}


\begin{frame}{Course Plan}
    \begin{itemize}
        \item Weeks 1 and 2: Introduction to Data Science, Python and Data Wrangling
        \item Week 3: Lab 1 - Python for Data Science and Data Wrangling
        \item Weeks 4 and 5: Linear Methods and Unsupervised Learning
        \item Week 6: Lab 2 - Linear methods and clustering
        \item Weeks 7 and 8: Kernel Methods, Trees and Neural Networks
        \item Weeks 9 and 10: Final Lab (Kaggle competition)
    \end{itemize}
\end{frame}


\begin{frame}{Instructors}
    \begin{itemize}
        \item Lecturer: Shyue Ping Ong (\href{mailto:ongsp@ucsd.edu}{ongsp@ucsd.edu})
        \item Teaching Assistant: Sojung Koo (\href{mailto:s2koo@ucsd.edu}{s2koo@ucsd.edu})
    \end{itemize}
\end{frame}


\begin{frame}
\frametitle{Recommended Textbooks (All Free)}

\begin{columns}
\column{0.5\textwidth}
\begin{figure}
    \centering
    \includegraphics[width=0.4\textwidth]{figures/book-cover-elements-of-statistical-learning.jpg}
    \caption{The Elements of Statistical Learning: Data Mining, Inference, and Prediction, Second Edition  \href{http://www.amazon.com/dp/0387848576/ref=cm_sw_em_r_mt_dp_U_LTvbEbPHKK8VF}{[Amazon]}\href{http://web.stanford.edu/~hastie/ElemStatLearn/}{[Free PDF]}}
\end{figure}
\column{0.5\textwidth}
\begin{figure}
    \centering
    \includegraphics[width=0.4\textwidth]{figures/book-cover-python-data-science.jpg}
    \caption{Python Data Science Handbook
        \href{https://www.amazon.com/dp/1491912057/ref=cm_sw_em_r_mt_dp_U_g0vbEb9N7HVQD}{[Amazon]} \href{http://jakevdp.github.io/PythonDataScienceHandbook/}{[Free web version]}}
\end{figure}
\end{columns}
\end{frame}


\begin{frame}{Course Structure}
    \begin{itemize}
        \item Lectures/Labs (Tues/Thurs @ 930-1050).
        \item Note: Please ignore scheduled lab sessions - all labs are held in lecture times, not in a separate time.
        \item Recordings will be available online after the class.
        \item \textbf{Please bring your laptops.}
        \item Grading:
        \begin{itemize}
            \item Lab 1: 25\% - Same for NANO181R and NANO281R
            \item Lab 2: 25\% - Same for NANO181R and NANO281R
            \item Lab 3: 50\% - Different for NANO181R and NANO281R
        \end{itemize}
    \end{itemize}
\end{frame}


\begin{frame}{Lab Assessment Criteria}
\begin{table}[]
    \centering
    \begin{tabular}{|c|c|}
    \hline
Model performance & 30\%\\
\hline
Materials Science Insights & 30\%\\
\hline
Data Science Technique & 30\%\\
\hline
Programming Style & 10\%\\
\hline
    \end{tabular}
\end{table}
\end{frame}

\begin{frame}{Class policies}
    \begin{itemize}
        \item \textbf{Collaboration}:  Working together is highly encouraged, but each student must submit his / her own work.
        \item \textbf{Use of AI}: Despite being an course in AI/ML techniques, use of AI tools such as ChatGPT or similar is \textbf{strongly discouraged}. You need to work through the exercises, including the mistakes and iterations, to learn the concepts. The instructors will use AI detection tools on your labs - we know what an AI-generated answer looks like. If the problem is severe enough, all credit for Labs 1 and 2 will be zeroed out, and all evaluation will be done only on the final lab, which is open-ended and therefore not amenable to AI cheating.
        \item \textbf{Remember that you are here to learn an essential skill}.
    \end{itemize}
\end{frame}


\begin{frame}{Class etiquette}
    \begin{itemize}
        \item Interaction preferred - interruptions with questions highly encouraged.
        \item Please be punctual. Lectures will start on time.
        \item Use of laptop to follow class examples is encouraged, but please be respectful of your lecturer and classmates by not using devices for non-class applications.  All devices must be on silent mode.
    \end{itemize}
\end{frame}


\begin{frame}{Prerequisites}
    \begin{itemize}
        \item Knowledge of basic statistics (e.g., Gaussian distribution, Bayes theorem, etc.).
        \item Knowledge of basic linear algebra (e.g., matrix multiplication, eigenvalue decomposition, inverse).
        \item Some programming experience. Ideally, experience in the Python programming language would be helpful.
        \item 1st Homework (ungraded):
        \begin{enumerate}
            \item Go to \href{https://colab.research.google.com/}{Google Colab}.
            \item Create a new notebook.
            \item Go through items 1-3 in the \href{http://docs.python.org/3/tutorial/}{official Python tutorial} – please run through the actual tutorial line by line. It should not take you more than 30 mins to do the whole thing.
            \item Extra: Briefly read through item 4 in the tutorial on flow control (if and for statements, especially).
        \end{enumerate}
    \end{itemize}
\end{frame}

\begin{frame}{Course Admin}
    \begin{itemize}
        \item Canvas for all course admin, including announcements/communications and submission of labs.
        \item \href{https://materialsvirtuallab.github.io/nano281/}{NANOx81 } for all materials, including labs with instructions.
        \begin{figure}
            \centering
            \includegraphics[width=0.25\textwidth]{figures/QR-NANO281Github.png}
            \label{fig:my_label}
        \end{figure}
    \end{itemize}
\end{frame}


\begin{frame}{Questions and Feedback}
    \begin{itemize}
        \item Questions welcomed at any time during or after lectures
        \item NANO181R/NANO281R is relatively very new - the instructors will try their best, but I would ask for you to be tolerant of any issues while we continue to improve the curriculum and labs.
        \item Your \textbf{feedback} is invaluable for shaping the current course as well as future courses.
        \item Email all feedback directly to \href{mailto:ongsp@ucsd.edu}{ongsp@ucsd.edu}.
    \end{itemize}
\end{frame}


\begin{frame}
    \Huge{\centerline{The End}}
\end{frame}

\end{document}
