\documentclass[aspectratio=169]{beamer}
\usepackage[utf8]{inputenc}
\usepackage{hyperref}
\usepackage{amsmath,amsfonts,amsthm,bm}
\usepackage{color}
\usepackage{minted}
\usepackage{graphicx} % Allows including images
\usepackage{booktabs} % Allows the use of \toprule, \midrule and \bottomrule in tables
\usepackage{tikz}
\usepackage[version=3]{mhchem}
\usepackage{pgfplots}
\pgfplotsset{compat=1.16} 
\setminted{fontsize=\scriptsize}

\hypersetup{
    colorlinks=true,
    linkcolor=red,
    filecolor=magenta,      
    urlcolor=red,
}

\DeclareMathOperator*{\argmax}{argmax}
\DeclareMathOperator*{\argmin}{argmin}
\let \vec \mathbf

\newcommand{\classname}{NANOx81R}
\newcommand{\classyear}{Spring 2025}
\mode<presentation> {
    \usetheme{CambridgeUS}
    \setbeamertemplate{footline}[text line]{%
      \parbox{\linewidth}{\vspace*{-8pt}\classname\hfill\classyear\hfill\insertpagenumber}}

    %\setbeamertemplate{footline}[page number]
    \setbeamertemplate{navigation symbols}{}
}


\title[Maximum Likelihood and Bayesian Methods]{Maximum Likelihood and Bayesian Methods}

\author{Shyue Ping Ong}
\institute[UCSD]{University of California, San Diego\\
\medskip
}
\date{\classyear} % Date, can be changed to a custom date

\begin{document}


\begin{frame}
    \titlepage % Print the title page as the first slide
\end{frame}


\begin{frame}{Overview}
    \tableofcontents
\end{frame}


\section{Preliminaries}

\begin{frame}{Preliminaries}
    \begin{itemize}
        \item Thus far, we have discussed ML models in the context of minimizing a loss function, typically a sum of squares.
        \item These approaches have a probabilistic interpretation and are instances of the maximum likelihood (ML) approach to fitting.
    \end{itemize}
\end{frame}

\section{Maximum Likelihood Inference}


\begin{frame}{Maximum Likelihood Inference}
    \begin{itemize}
        \item Consider a probability distribution/mass function for the observations:
        \begin{equation*}
            z_i \sim g_{\theta}(z)
        \end{equation*}
        \item Also, known as a parametric model for $Z$, with $\theta$ being unknown parameter that govern the distribution. 
        \item Let us assume that $Z$ has a normal distribution ($\theta = (\mu, \sigma^2)$):
        \begin{equation*}
            g_{\theta} = \frac{1}{\sqrt{2\pi}\sigma} e^{-\frac{(z-\mu)^2}{2\sigma^2}}
        \end{equation*}
        \item The likelihood of the observed data under model $g_{\theta}$ is then given by the \textit{likelihood function}:
        \begin{equation*}
            L(\theta; \vec{Z}) = \prod_{i=1}^N g_{\theta}(z_i)
        \end{equation*}
    \end{itemize}
\end{frame} 


\begin{frame}{Maximum Likelihood Inference, cont.}
    \begin{itemize}
        \item Consider the log of $L$:
        \begin{equation*}
            \ell(\theta; \vec{Z}) = \sum_{i=1}^N \log{g_{\theta}(z_i)}
        \end{equation*}
        \item \textit{Score function} $\dot{\ell}(\theta; \vec{Z}) = 0$ at maximum (dot means derivative).
        \item \textit{Information matrix}:
        \begin{equation*}
            \vec{I}(\theta) = -\sum_{i=1}^N\frac{\partial^2 \ell(\theta;z_i)}{\partial \theta \partial \theta^T}
        \end{equation*}
        \item \textit{Observed information} = $\vec{I}(\hat{\theta})$ ($\hat{\theta}$ is the estimated parameter)
        \item \textit{Fisher information} $\vec{i}(\theta) = E_{\theta}[\vec{I}(\theta)]$, widely used in optimal experimental design.
    \end{itemize}
\end{frame}


\begin{frame}{Linear Regression Revisited as MLE}
    \begin{itemize}
        \item $Y = \beta_0 + \beta_1 X + \varepsilon$
        \item $\varepsilon \sim N(0, \sigma^2)$, i.e., independent Gaussian noise.
        \begin{eqnarray*}
        L(\beta_0, \beta_1, \sigma; \vec{y}) &= & \prod_{i=1}^N P(y_i|x_i;\hat{\beta_0},\hat{\beta_1}, \hat{\sigma})\\
        & = &\prod_{i=1}^N \frac{1}{\hat{\sigma}\sqrt{2\pi}} e^{-\frac{(y_i-\hat{\beta_0} - \hat{\beta_1}x_i)^2}{2\hat{\sigma}^2}}\\
        \ell(\beta_0, \beta_1, \sigma; \vec{y})& = & \sum_{i=1}^N \log{\frac{1}{\hat{\sigma}\sqrt{2\pi}} e^{-\frac{(y_i-\hat{\beta_0} - \hat{\beta_1}x_i)^2}{2\hat{\sigma}^2}}}\\
        & = & -\frac{N}{2}\log{\hat{\sigma^2}2\pi} - \sum_{i=1}^N \frac{(y_i-\hat{\beta_0} - \hat{\beta_1}x_i)^2}{2\hat{\sigma}^2}
        \end{eqnarray*}
        \item Clearly, MLE in this case is equivalent to minimizing least squares.
    \end{itemize}
\end{frame}


\section{Bayesian Methods}


\begin{frame}{Bayesian Methods}
    \begin{itemize}
        \item \textit{Posterior} distribution given by:
        \begin{equation*}
            P(\theta|\vec{Z}) = \frac{P(\vec{Z}|\theta)P(\theta)}{\int{P(\vec{Z}|\theta)P(\theta)}d\theta}
        \end{equation*}
        \item Updated knowledge about $\theta$ after seeing data.
    \end{itemize}
\end{frame}


\section{Expectation Maximization}


\begin{frame}{Expectation Maximization}
    \begin{itemize}
        \item Popular approach for solving MLE problems.
        \item Algorithm (example for two-component Gaussian mixture):
        \begin{enumerate}
            \item Start with initial guesses for parameters (e.g, two random $y_i$).
            \item Expectation step: Compute the \textit{responsibilities}.
            \begin{equation*}
                \hat{\gamma_i} = \frac{\hat{\pi}\phi_{\hat{\theta_2}}(y_i)}{(1-\hat{\pi})\phi_{\hat{\theta_1}}(y_i) + \hat{\pi}\phi_{\hat{\theta_2}}(y_i)}, i=1,2,...,N
            \end{equation*}
            \item Maximization step: Compute weighted means and variances.
            \begin{eqnarray*}
            \hat{\mu_1}& = \frac{\sum_{i=1}^N (1-\hat{\gamma_i})y_i}{\sum_{i=1}^N (1-\hat{\gamma_i})}, \hat{\sigma_1^2}& = \frac{\sum_{i=1}^N (1-\hat{\gamma_i})(y_i-\hat{\mu_1})^2}{\sum_{i=1}^N (1-\hat{\gamma_i})}\\
            \hat{\mu_2}& =  \frac{\sum_{i=1}^N \hat{\gamma_i}y_i}{\sum_{i=1}^N \hat{\gamma_i}}, \hat{\sigma_2^2}& = \frac{\sum_{i=1}^N \hat{\gamma_i}(y_i-\hat{\mu_2})^2}{\sum_{i=1}^N \hat{\gamma_i}}
            \end{eqnarray*}
            with mixing probability $\hat{\pi} = \sum_{i=1}^N \hat{\gamma_i}/N$.
            \item Iterate until convergence.
        \end{enumerate}
    \end{itemize}
\end{frame}


\begin{frame}[fragile]{Gaussian mixture using EM in scikit-learn}
\inputminted{python}{example_sklearn_em.py}
\end{frame}


% \begin{frame}{Bibliography}
%     \bibliographystyle{unsrt}
%     \bibliography{refs}
% \end{frame}


\begin{frame}
    \Huge{\centerline{The End}}
\end{frame}

\end{document}

